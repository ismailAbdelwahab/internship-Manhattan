\chapter{Metrics}
\section{Definition}%%%%%%%%%%%%%%%%%%%%%%%%%%%%%%%%%%%%%%%%%%%%%%%%%%%%%
A metric is a function, more precisely a distance function. It is used to define how do wethe distance between two elements in a set.

This function takes as input two elements from a set and outputs the distance between them. Of course their are many ways to calculate a distance between two elements based on what you want and what type of data you are working on.

For a classic presentation we will introduce the l2-metric. Later in this report we will use the l1-metric as it is the one that is important for us.
\section{Example of metrics}%%%%%%%%%%%%%%%%%%%%%%%%%%%%%%%%%%%%%%%%%%%%%%%%
\subsection{The l2-metric}%%%%%%%%%
For the sake of simplification we will just present you the l2-metric, which is the one that we use in the "all days" world.

In fact, the \textbf{l2-metric} is also know as the \textbf{Euclidean distance} and uses the formula relative to the Pyhtagorean theorem:
	\[d_2(p,q)= d_2(q,p) = ||p - q|| = \sqrt{(x_q-x_p)^2+(y_q-y_p)^2}\]
	
This is the casual way to calculate distances on a map or between points.
\subsection{The l1-metric}%%%%%%%%%%
	And now as simple as it is, the \textbf{l1-metric} is just another way to calculate the distance between two points p and q:
	\[ d_{1}(p,q)= d_{1}(q,p) = |x_p-x_q|+|y_p-y_q|\]
What is interesting in this metric here, is that we calculate a distance based of the variation of x coordinates and the variation of y coordinates.

Which means that we give some importance to the fact that we can only create edges on those two axes. This will be really useful later when we will speak about Pareto envelopes and minimal Manhattan network in more details.