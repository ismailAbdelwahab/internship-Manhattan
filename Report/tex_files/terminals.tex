\chapter{Terminals}
\section{Points in the space}
Points in the space are named \textbf{terminals}, in this paper a terminal will refer to a point in the space.\newline

For our study, we will focus ourselves on two dimensional space, but keep in mind that with a little more work and a lot of adjustments, all of what you are going to discover can be applied to more dimensions.

A point will be defined with Cartesian coordinates. In this paper we will note terminals with lower case letters. We will talk of a point \textbf{\emph{p}} or a point \textbf{\emph{q}} for example.

When speaking about one of the coordinate of a point we will use subscript notation as: $x_p$ for the x coordinate of the point \emph{p}.

\section{Terminal's properties} %%%%%%%%%%%%%%%%%%%%%%%%%%%%%%%%%%%%%%%%%%%%%
Before going further in the subject, we will need to define some properties of our terminals.

These properties are the core of our computation as we will use them in order to run some algorithms over the set of terminals. As it is of high importance, the definition of these properties are given here. 
\subsection{Domination} %%%%%%%%%%%%%%
Let T bet a cloud of points in two dimensions.

Let \emph{p} and \emph{q} be two terminals of T.

\noindent It is said that \emph{p} dominates \emph{q} \underline{if and only if}:
\begin{enumerate}[noitemsep, nolistsep]
	\item{$\forall t \in T : d(p,t) \leq d(q,t)$}
	\item{$\exists t' \in T : d(p,t) < d(q,t)$}
\end{enumerate} 
With d(p,t) the distance between \emph{p} and \emph{t}, that is calculated based on the metric that we use. The next whole chapter is dedicated to, and will explain, metrics and how do we calculated distances.

In this definition we need to note that no point can be dominated. As for all \emph{t} in T d(p,t) must be lower or equal to d(q,t), (of course we take in consideration that \emph{p} and \emph{q} are different points) then \emph{t} could be \emph{q}. In this case then it is impossible to have $d(p,q)\leq d(q,q)$.

In any case if a terminal \emph{p} dominates another terminal \emph{q} we denote this property by: $ p \succ q $
\subsection{Efficiency}%%%%%%%%%%%%%%%%
A point \emph{p} is said to be efficient if $\neg\exists q$ and \emph{q} dominates \emph{p}. In other words: a terminal is said efficient if no other terminal is dominating it.

As no points should be dominated, in fact all terminals in our cloud of points are efficient terminals.
\subsection{Equivalence} %%%%%%%%%%%%%%
Let A be a subset of  our space $\mathbb{R}^m$ : $A \subset \mathbb{R}^m$.

Let T be a cloud of points in $\mathbb{R}^m$: $T \in \mathbb{R}^m$.

Let \emph{p} and \emph{q} be two points of T: $p \in T$ and $q \in T$.

\noindent Our two points, \emph{p} and \emph{q} are said to be equivalent by A \underline{if and only if}:
\begin{itemize}[noitemsep, nolistsep]
	\item{$\forall a \in A : d(p,a) = d(q,a)$}
\end{itemize} 

So as we see here, for a part of our space we can define two points as equivalent if and only if for all terminals in this part, our two points are equidistant to each and every point in this part.

We denote this property as: $p \simeq_{A} q$ if p and q are equivalent by A.

\section{A set of terminals to begin with}%%%%%%%%%%%%%%%%%%%%%%%%%%%%%
To work on finding a MMN, we have to understand on what data we are searching this network. The objective of such a network is to \textbf{connect each an every possible pairs of points} in a set of terminals (in our case on the plane).

In fact connecting all possible pairs of terminals given a list of terminals is not difficult at all (for example just create the complete grid of your set of terminals: 

\begin{algorithm}[H]
\SetAlgoLined
 \caption{Complete grid from a set of Terminals "T"}
\KwResult{The complete grid of a set of terminals in 2D }
 lowX = lowest x value of all terminals\;
 highX = highest x value of all terminals\;
 lowY = lowest y value of all terminals\;
 highY = highest y value of all terminals\;
 completeGrid = [ An empty list of lines ]\;
 \For{Every terminal t in our set of terminals}{
 Add to completeGrid the line from (lowX,$y_t$) to (highX,$y_t$)\;
 Add to completeGrid the line from ($x_t$,lowY) to ($x_t$,highY)\;
 }
 return completeGrid\;
\end{algorithm}

Here we can see that collecting the minimal and maximal x and y coordinates, is a linear operation and the creation of both lines are also linear operations. In conclusion a naive Manhattan network can be found in O(n). 

However, generally this naive answer to our problem is far from the minimal Manhattan network. And here is the trick, this is what we want a \textbf{\emph{Minimal}} Manhattan network.
