\chapter{Internship proceeding}
This chapter is here to summarize quickly how did the internship went on, and what were the encountered difficulties.
\section{Covid-19 situation}
Due to the current situation the internship was made from my own home, as the confinement restriction didn't allowed me to work directly at the laboratory or at any library.

In this respect, the internship was for me a little harder as communicating with my internship mentors via e-mails was slower than asking direct questions to them. However they never missed on answering my questions and with a really helpful accuracy.

Unfortunately, due to this situation the internship was slowed and it was harder for me to work. I personally think that working from home is fine but not optimal, as the environment is not meant for working is would rather have worked directly at the laboratory to be more productive and efficient.

I hope that to a certain extent, you understand my position and take in consideration the fact that my work was greatly slowed by this situation.
\section{Encountered difficulties}
Most of the difficulties that I encountered was the consequence of the lack of knowledge on some specific topic. I was just stuck sometimes in my work, while calculating the Pareto envelope. Hopefully after a few mails to my internship's mentors, they redirected me to research papers and publication where all of what I needed was explained.

Also during video-conferences they explained me a lot of the properties hold by Pareto envelopes, strips and staircases.\newline

However during the internship we had a lot of final exams and numerous other semester's projects to submit to our professors.

 This took a lot of the work time that I could have added to this internship. A lot of work had to be done in the same as this internship and managing our time was really hard for us. I hope that in view of this situation you understand that I did my best and produced the best work I could for this internship while also trying to get the best marks by additionally giving the best of me for the other projects and exams too.

\section{Current status of the work}
In terms of code, my strategy was to code by implementing tests first. This technique was tough to me last year and it proved it's effectiveness again here.

As a consequence coding all tests before implementing my own code was a big guideline for me to know where I am in my work. This internship also tough me how to organize myself in the regard of a project in order to have clear and neat code.\newline 

My aim was to reach a point where i can compute a minimal Manhattan network, which is on a good way to be done. For now i did all the work needed to recognize Pareto envelope and to get the strips and staircases set.

The last part of my work which will be done in the next three weeks is to initiate myself to linear programming (for personal knowledge), and lastly in order to compute a MMN I need to implement a 2.5 approximating algorithm that does it([2]).\newline

I was hopping to finish all of my work before presenting you this report so that I could completely explain all of my work, but due to the current situation this was too hard to do. Hopefully my internship is giving me, I think, enough time to do the rest of my work.

\section{Continuation of the internship}
Finally i have to say that my internship is not finished and I will continue to work from home during approximately three to four weeks which will be enough to finish my work on this internship.

As this report is a manifest of the end of the matter "Stage ou projet - S6", I would like to thank some people before ending this report. 