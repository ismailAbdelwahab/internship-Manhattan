\chapter{Minimal Manhattan networks (MMN)}
\section{Discoverer of the problem}%%%%%%%%%%%%%%%%%%%%%%%%%%%%%%%%%%%%%%%%%
Gudmundsson, Levcopoulos and Narasimhan are three scientists that work together in the 90's on a specific topic, Finding a Minimal manhattan network for a given set of terminals. In fact a the time it wasn't know that this problem was actually NP-Hard, and required a lot of calculation to be solved.

This is why after a long work the pionier of the MMN publish two algorithms with factors 4 and 8. Knowing that this problem was bind to be difficult they emmited the conjecture that a factor 2 algorithm must exist.

And it is at this time that they opened the door to a vaste community of researchers to work on this topic and variants of this problem.

But before going deeper into the explanation we need to define the elements that we are going to work on. 

\section{Origins of the topic}%%%%%%%%%%%%%%%%%%%%%%%%%%%%%%%%%%%%%%%%%%%%%%%
This problem was formulated in 1999 by Gudmundsson, Levcopoulos and Narasimhan whom found two approximating algorithms of factors 4 and 8. 

Moreover they conjectured the existance of an approximating algorithm of factor 2. Which means that this conjectured algorithm could give the solution to our problem with a cost lower than two times the optimal cost to find the optimal solution.

This algorithm was found in 2005 by my internship's mentors alon with K.Nouioua and N.Catusse by using rounding method on a factional optimal solution, and then coupling it with a primal-dual algorithm.

Later other variants of this problem were studied, like for example the generalization of the problem to the normed plane with polygonal balls (in the l1-metric this polygon ball is a lozenge). And for this version of the MMN an algorithm factor 2.5 was found [2].

\section{What is a rounding algorithms}%%%%%%%%%%%%%%%%%%%%%%%%%%%%%%%%%%



\section{A set of terminals to begin with}%%%%%%%%%%%%%%%%%%%%%%%%%%%%%
To work on finding a MMN, we have to understand on what data we are searching this network. The objective of such a network is to \textbf{connect each an every possible pairs of points} in a set of points (on the plane in our case).

These points are called \textbf{terminals}.

In fact connecting all possible pairs of terminals given a list of terminals is not difficult at all (for example just create the complete grid of your set of terminals: 

\begin{algorithm}[H]
\SetAlgoLined
 \caption{Complete grid from a set of Terminals "T"}
\KwResult{The complete grid of a set of terminals in 2D }
 lowX = lowest x value of all terminals\;
 highX = highest x value of all terminals\;
 lowY = lowest y value of all terminals\;
 highY = highest y value of all terminals\;
 completeGrid = [ An empty list of lines ]\;
 \For{Every terminal t in our set of terminals}{
 Add to completeGrid the line from (lowX,$y_t$) to (highX,$y_t$)\;
 Add to completeGrid the line from ($x_t$,lowY) to ($x_t$,highY)\;
 }
 return completeGrid\;
\end{algorithm}

Here we can see that collecting the minimal(resp. maximal) x(resp. y) is a linear opération and the creation of both lines are also linear operations. In conclusion a simple Manhattan network can be found in O(n).

But here is the trick, what we want is a \textbf{\emph{Minimal}} Manhattan network.

\section{The l1-metric}%%%%%%%%%%%%%%%%%%%%%%%%%%%%%%%%%%%%%%%%%%%%%%%%%%%%%
We can briefly define  a \textbf{metric} as "\emph{a way to calculate the distance between two points}".

For the sake of simplification I will just present you the l2-metric, which is the one that we use "all days".

The \textbf{l2-metric} also know as the \textbf{Euclidean distance} uses the formula based on the Pyhtagorean theorem:
	\[d_2(p,q)= d_2(q,p) = ||p - q|| = \sqrt{(x_q-x_p)^2+(y_q-y_p)^2}\]
	
And now as simple as it is, the \textbf{l1-metric} is just another way to calculate the distance between two points p and q:
	\[ d_{1}(p,q)= d_{1}(q,p) = |x_p-x_q|+|y_p-y_q|\]

\section{Introduction to the MMN}%%%%%%%%%%%%%%%%%%%%%%%%%%%%%%%%%%%%%%%%%
A Minimal Manhattan network has obviously the same properties as a Manhattan network, but we add to them a constraint: the sum of the length of all edges (lines in the network) must be the lowest possible.

It is by using numerous algorithms and data structures that we will ensure that what we create is a real minimal Manhattan network.

This simple constraint will totally change the way we approach the problem and how shall we solve it. By trying to find  a minimal solution we are greatly increasing the complexity of the work that will be needed to answer our question.

\section{Pareto envelopes}%%%%%%%%%%%%%%%%%%%%%%%%%%%%%%%%%%%%%%%%%%%%%%%%%%%

\section{Finding a MMN}%%%%%%%%%%%%%%%%%%%%%%%%%%%%%%%%%%%%%%%%%%%%%%%%%%%%%%