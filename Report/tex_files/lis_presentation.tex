\chapter{Presentation of the Laboratory}
\section{Creation of the structure}
The Computer Science and Systems Laboratory (LIS) in french "Laboratoire d’Informatique et Systèmes", is a structure that is the resulting of the merge of two laboratories. LIS was created from The "Laboratoire Fondamentale de Marseille" (LIF) in conjunction with the "Laboratoire des Sciences de l'Information et des Systèmes (LSIS).

\section{Supervisors and partner of the LIS}
LIS is a research lab which is put under the supervision of the "Centre National de la Recherche Scientfique" (CNRS), Aix-Marseille University and University of Toulon. 

The LIS also has as a partener the "Ecole Centrale de Marseille" (ECM).

\section{Localization of the laboratory}
The Computer Science and Systems Laboratory is located in France and is split upon two cities, Marseille and Toulon. On Marseille we can find two places that are used by the LIS: the first one being on the university campus of Saint-Jérôme and the other on Luminy. Also on Toulon the campus of the University of Toulon is where the LIS operates.
 
\section{Workforce}
Spread over all of theses locations, the LIS is represented by 375 members from which we can count:
\begin{itemize}[noitemsep, nolistsep]
	\item{190 Tenure researchers and professors}
	\item{125 Doctoral students}
	\item{40 Post-docs }
	\item{20 Technical staff}
\end{itemize} 

\section{Reason behind the creation of the LIS}
The Computer Science and Systems Laboratory was created so that we can use the strengths and competences of both LSIS and LIF together in order to meet and overcome new scientific challenges.

\section{Hierarchy of the laboratory}
While keeping this objective in head this laboratory created four departments. Each of them will inherit and focus their research on a specific domain that was previously studied by the LSIS, the LIF or both of them.
Those four departments are:
\begin{itemize}[noitemsep, nolistsep]
	\item{Calculus}
	\item{Data science}
	\item{Signal and image}
	\item{System Analysis and Automatic Control}
\end{itemize}
Also each department is divided into subgroups working on specific topic treated by their department.

\subsection{Calculus department}
This department of the LIS was created with the objective to focus on some specific research activities like: Theoretical computer science, logic, obviously algorithmic and complexity, geometric and topology, but also quantum computing and artificial intelligence.

My internship was done with a subgroup of this department in Marseille - Luminy.
\subsection{Data science department}
Nowadays we know the importance of data, should it be founding it, treating data or even analyzing it. 

So at the LIS a department is dedicated to the science of data, they are working on  machine learning which is a growing technology, natural language processing that is also an upcoming technology that will be vastly used, data mining and information retrieval to collect data which is as important as computing it, and of course artificial intelligence that work hand in hand with machine learning.
\subsection{Signal and image department}
The art of manipulating visual data is represented by this department. As showing information is as important as gathering it a whole department is dedicated to this.

They especially work on  image processing to treat images an recover data from them. But also on audio and bio-signal processing, medical imaging and here we see the importance of this department along the fact that they work with the medical field. And lastly image modeling is a topic that this department is competent to treat.
\subsection{System Analysis and Automatic Control (ACS) department}
The last department to be presented is making research on control theory, diagnostic, decision theory, system simulation and modeling too. 

This department is focus on the analysis of systems, their correctness and also the automatic controls that can be made over them.