\chapter{Pareto envelopes}

\section{Definition of Pareto envelopes}%%%%%%%%%%%%%%%%%%%%%%%%%%%%%%%%%%
\subsection{Simple definition}
	The Pareto envelope of a set of terminals is a structure that is defined by a subset of terminals constructing this set. This subset is formed with only all the non-dominated points in the space.
	
\subsection{Remarkable result}
	If we study the the Pareto envelope and the convex hull(also known as convex envelope) of a cloud of points we see that they coincide but only if we use the euclidean distance (l2-metric).
	
	This result is important as it make us understand that \textbf{convex envelopes} and \textbf{Pareto envelopes} are some kind of linked but \textbf{not the same object}.
	
\section{Calculation with $I_d$ in $\mathbb{R}^2$}%%%%%%%%%%%%%%%%%%%%%%%
\subsection{A set of efficient points: $\Upsilon_d$}%%%%%%%%%%%%%%%%%%%%%%%%%%%%%%%%%%%%%%%%%
	Before calculating the Pareto envelope we need to be introduced, referring to a set of terminals, to the set of efficient points that is linked to it.
	
	This set is determined by the intervals that we calculated previously. So it is natural that this set is totally depending on the metric we are working with.
	
	It is in that matter that we have to know that the Pareto envelope depends also on the metric used to calculate distances. 

\subsection{Construction of $\Upsilon_d$} %%%%%%%%%
\noindent This set is noted $\Upsilon_d$ and is defined as:
\begin{equation}
	\Upsilon_d = \cap^{n}_{i=1}(\cup^{n}_{j=1}I_{d}(t_i,t_j))
\end{equation}

As you can see the equation 6.1 stipulates that the set of efficient points is defined by the intersection of all intervals from all points to all others. Meaning that $\Upsilon_d$ is defined by the intersection of all intervals with all pairs of points possible.

And it is with this definition that we see that we are getting closer and closer to construct our target, a minimal Manahattan network.

\subsection{$\Upsilon_d$ in $\mathbb{R}^2$}%%%%%%%%%%
With this tool we can now construct the set of efficient point in our cloud of points in $\mathbb{R}^m$.

As we are for now working a specific space ($\mathbb{R}^2$), we can now specifically construct $\Upsilon_{d_1}$ for the l1-metric and $\Upsilon_{d_{\infty}}$ in the L-infinity metric.

\section{Different behavior in $\mathbb{R}^3$}%%%%%%%%%%%%%%%%%%%%%%%%%%%
This remark is here for the continuation of this intern ship and will be useful for later. 

We have to be careful when working on higher dimension and to not generalize our result too quickly.

In fact as for $\mathbb{R}^2$, the Pareto envelope is $\Upsilon_{d_1}$ or $\Upsilon_{d_\infty}$ depending on the metric used, it is not the case in $\mathbb{R}^3$.

In $\mathbb{R}^3$ the Pareto envelope is more complex to calculate and moreover differs from $\Upsilon_d$ shall it be with euclidean distances, l1-metric or even l-infinite metric.

As the Pareto envelope is calculated in $O(n log(n))$ in $\mathbb{R}^2$ due to this difference in $\mathbb{R}^3$, the Pareto envelope in $\mathbb{R}^3$ is not that easy to compute.

Again as we are working in $\mathbb{R}^2$ we are going to discover an algorithm designed to calculate the Pareto envelope in such a space in $O(nlog(n))$ as it is the lowest complexity to do so.

\section{Construction of the Pareto envelope with a scanning algorithm}%%%%%%%%%
\subsection{Dividing the problem into chains}
The easiest way to construct the Pareto envelope is to divide the problem into four chains. For that we will construct the chains $S1$,$S2$,$S3$,$S4$. Clockwise the chains are built in this manner:
\begin{itemize}[noitemsep, nolistsep]
	\item{$S1$ - North to East}
	\item{$S2$ - East to South}
	\item{$S3$ - South to West}
	\item{$S4$ - West to North}
\end{itemize}

After computing all four chains, concatenate them in the right order $S1 -> S4$ and this data structure will represent the Pareto envelope. 

\subsection{Computing the chains}
Here is an algorithm to calculate the first of the four chains.

\begin{algorithm}[H]
\SetAlgoLined
 \caption{Compute chain S1}
\KwResult{The S1 chain from a could of terminals "T"}
 Sort the terminals by decreasing ordinate and increasing abscissa.\;
 Let T be the set of terminals.( T = [$t_0$, $t_1$, ..., $t_{n-1}$])\;
 S1 = \o\;
 previous = $t_0$\;
 \For{i=2 to (n-1)}{
 	\uIf{$x_{t_i} > x_{previous}$ AND $y_{t_i} == y_{previous}$}{
 		$ S1 = S1 \cup [previous, t_i]$\;
 		$previous = t_i$\;
 	}
 	\uIf{$x_{t_i} > x_{previous}$ and $y_{t_i} > y_{previous}$}{
 		temp = ($x_{previous},y_{t_i}$)
 		$S1 = S1 \cup [previous,temp] \cup [temp,t_i]$\;
 		$previous = t_i$\;
 	}
 }
 return S1\;
\end{algorithm}

Here S1 is a list of points, these points represent the "breaking points" along the chain. Drawing this chain can be done by drawing a line from a point to his successor in the list.\newline

\textbf{Remark:} The construction of S2,S3 and S4 is done in the same way, the only thing to change are:
\begin{enumerate}[noitemsep, nolistsep]
	\item{The sort applied to T.}
	\item{The conditions in the two "if" statements.}
	\item{The name of the returned variable.}
\end{enumerate}

\subsection{Complexity}
Despite doing this algorithm four time, the overall complexity (construction of the four chains) is $O(nlog(n))$.\newline

The first step in this algorithm, is to sort the points. Obviously we use an $O(nlog(n))$ algorithm for that as it is the quickest way to do so.

All other instruction are made in linear time. We only insert points into a list which is a linear timely done and verify condition (also in linear time).
\begin{itemize}[noitemsep, nolistsep]
	\item{Line 1 // Sorting the terminals // $O(nlog(n))$}
	\item{Line 2 to 4 // Initialization // $O(1)$}
	\item{Line 5 to 13 // For loop // $O(n)$}
\end{itemize}

\noindent The complexity of this algorithm is:

$O(nlog(n)) * O(1) * O(n) = O(nlog(n))$