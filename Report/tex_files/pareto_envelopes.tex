\chapter{Pareto envelopes}

\section{Definition of Pareto envelopes}%%%%%%%%%%%%%%%%%%%%%%%%%%%%%%%%%%
\subsection{Simple definition}
	The Pareto envelope of a set of terminals is a structure that is defined by a subset of terminals constructing this set. This subset is formed with only all the non-dominated points in the space.
	
\subsection{Remarkable result}
	If we study the the Pareto envelope and the convex hull(also known as convex envelope) of a cloud of points we see that they coincide but only if we use the euclidean distance (l2-metric).
	
	This result is important as it make us understand that \textbf{convex envelopes} and \textbf{Pareto envelopes} are some kind of linked but \textbf{not the same object}.
	
\section{Calculation with $I_d$ in $\mathbb{R}^2$}%%%%%%%%%%%%%%%%%%%%%%%
\subsection{A set of efficient points: $\Upsilon_d$}%%%%%%%%%%%%%%%%%%%%%%%%%%%%%%%%%%%%%%%%%
	Before calculating the Pareto envelope we need to be introduced, referring to a set of terminals, to the set of efficient points that is linked to it.
	
	This set is determined by the intervals that we calculated previously. So it is natural that this set is totally depending on the metric we are working with.
	
	It is in that matter that we have to know that the Pareto envelope depends also on the metric used to calculate distances. 

\subsection{Construction of $\Upsilon_d$} %%%%%%%%%
\noindent This set is noted $\Upsilon_d$ and is defined as:
\begin{equation}
	\Upsilon_d = \cap^{n}_{i=1}(\cup^{n}_{j=1}I_{d}(t_i,t_j))
\end{equation}

As you can see the equation 6.1 stipulates that the set of efficient points is defined by the intersection of all intervals from all points to all others. Meaning that $\Upsilon_d$ is defined by the intersection of all intervals with all pairs of points possible.

And it is with this definition that we see that we are getting closer and closer to construct our target, a minimal Manahattan network.

\subsection{$\Upsilon_d$ in $\mathbb{R}^2$}%%%%%%%%%%
With this tool we can now construct the set of efficient point in our cloud of points in $\mathbb{R}^m$.

As we are for now working a specific space ($\mathbb{R}^2$), we can now specifically construct $\Upsilon_{d_1}$ for the l1-metric and $\Upsilon_{d_{\infty}}$ in the L-infinity metric.

\section{Different behavior in $\mathbb{R}^3$}%%%%%%%%%%%%%%%%%%%%%%%%%%%
This remark is here for the continuation of this intern ship and will be useful for later. 

We have to be careful when working on higher dimension and to not generalize our result too quickly.

In fact as for $\mathbb{R}^2$, the Pareto envelope is $\Upsilon_{d_1}$ or $\Upsilon_{d_\infty}$ depending on the metric used, it is not the case in $\mathbb{R}^3$.

In $\mathbb{R}^3$ the Pareto envelope is more complex to calculate and moreover differs from $\Upsilon_d$ shall it be with euclidean distances, l1-metric or even l-infinite metric.

As the Pareto envelope is calculated in $O(n log(n))$ in $\mathbb{R}^2$ due to this difference in $\mathbb{R}^3$, the Pareto envelope in $\mathbb{R}^3$ is not that easy to compute.

Again as we are working in $\mathbb{R}^2$ we are going to discover an algorithm designed to calculate the Pareto envelope in such a space in $O(nlog(n))$ as it is the lowest complexity to do so.

\section{Construction of the Pareto envelope with a scanning algorithm}%%%%%%%%%
\subsection{Subdividing the problem into chains}
\subsection{Computing the chains}
\subsection{Scanning algorithm}
\subsection{Resulting envelope}